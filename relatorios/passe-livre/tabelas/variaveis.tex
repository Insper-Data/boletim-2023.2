\begin{table}
    \centering
    \caption{Variáveis utilizadas nos modelos econométricos}

    \begin{tabular}{m{0.18\linewidth}|m{0.7\linewidth}}
        Variável & Descrição \\
        \toprule
        Tratamento & Variável binária que assume 1 caso o município tenha recebido passe livre e 0 caso contrário \\
        \midrule
        Competitividade & Quão proximos foram os resultados entre primeiro e segundo candidatos. Calculado por $1/[\log({V_A/V_B})]$, na qual $V_A$ representa o número de votos recebidos pelo candidato mais votado e $V_B$, pelo segundo candidato mais votado.\footnotemark \\
        \midrule
        População & População do município \\
        \midrule
        PIB per capita & PIB per capita do municipio, disponível até 2020. Para 2022, foram utilizados os últimos dados disponíveis, de 2020. \\
        \midrule
        Beneficiados & Número de pessoas no municipio dividido pela quantidade de eleitores aptos. \\
        \midrule
        IDEB & Nota da educacao dos anos finais do ensino fundamental nas escolas publicas. A nota é apenas calculadas nos anos ímpares, com o primeiro dado disponível em 2005, então foram utilizados dados defasados em um ano. \\
        \midrule
        PIB governo & É definido como valor adicionado bruto a preços correntes da administração, defesa, educação e saúde públicas e seguridade social dividido pelo PIB municipal. \\
        \midrule
        Eleitores por seção & Média do número de eleitores aptos por seção eleitoral no município

    \end{tabular}

    \label{tab_variaveis}
\end{table}

\footnotetext[\thefootnote]{competitividade calculada pós}

% \begin{table}
%     \centering
%     \begin{tabular}[t]{llrrrrrr}
%     \toprule
%     \multicolumn{2}{c}{ } & \multicolumn{3}{c}{Primeiro Turno} & \multicolumn{3}{c}{Segundo Turno} \\
%     \cmidrule(l{3pt}r{3pt}){3-5} \cmidrule(l{3pt}r{3pt}){6-8}
%       & Passe Livre & Mean & SD & N & Mean & SD & N\\
%     \midrule
%     Abstenção & Não Houve & \num{0.21} & \num{0.05} & 5437 & \num{0.21} & \num{0.05} & 5148\\
%      & Houve & \num{0.20} & \num{0.03} & 81 & \num{0.20} & \num{0.03} & 370\\
%     \bottomrule
%     \end{tabular}
%     \caption{Abstenção e tamanho amostral para cada grupo em 2022}
%     \label{tab_descritiva}
%     \end{table}